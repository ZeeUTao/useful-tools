%pdfLatex

\documentclass[10pt,a4paper]{report}

\usepackage{tikz}
\usepackage{braids}
\usepackage{xifthen}

\usepackage{lastpage}
\usepackage[section]{placeins}
\usepackage[hidelinks,unicode=true]{hyperref}
\usepackage[capitalise]{cleveref}


\usepackage{graphicx}% Include figure files
\usepackage{dcolumn}% Align table columns on decimal point
\usepackage{bm}% bold math
\usepackage{epstopdf}
\usepackage{epsfig}
\usepackage{amsfonts}
\usepackage{graphicx}
\usepackage[center]{titlesec}
\usepackage{amsmath} % AMS Math Package
\usepackage{amsthm} % Theorem Formatting
\usepackage{amssymb}	% Math symbols such as \mathbb
\usepackage{float}
\usepackage{textcomp}
\usepackage{mathrsfs} 

\newcommand{\sxf}[4]{ X_{#1} X_{#2}X_{#3} X_{#4} }
\newcommand{\szf}[4]{ Z_{#1}Z_{#2}Z_{#3} Z_{#4} }
\renewcommand{\labelenumi}{(\alph{enumi})} % Use letters for enumerate
% \DeclareMathOperator{\Sample}{Sample}
\let\vaccent=\v % rename builtin command \v{} to \vaccent{}
\renewcommand{\v}[1]{\ensuremath{\mathbf{#1}}} % for vectors
\newcommand{\gv}[1]{\ensuremath{\mbox{\boldmath$ #1 $}}} 
% for vectors of Greek letters
\newcommand{\uv}[1]{\ensuremath{\mathbf{\hat{#1}}}} % for unit vector
\newcommand{\abs}[1]{\left| #1 \right|} % for absolute value
\newcommand{\avg}[1]{\left< #1 \right>} % for average
\let\underdot=\d % rename builtin command \d{} to \underdot{}
\renewcommand{\d}[2]{\frac{d #1}{d #2}} % for derivatives
\newcommand{\dd}[2]{\frac{d^2 #1}{d #2^2}} % for double derivatives
\newcommand{\pd}[2]{\frac{\partial #1}{\partial #2}} 
% for partial derivatives
\newcommand{\pdd}[2]{\frac{\partial^2 #1}{\partial #2^2}} 
% for double partial derivatives
\newcommand{\pdc}[3]{\left( \frac{\partial #1}{\partial #2}
	\right)_{#3}} % for thermodynamic partial derivatives
\newcommand{\ket}[1]{\left| #1 \right>} % for Dirac bras
\newcommand{\bra}[1]{\left< #1 \right|} % for Dirac kets
\newcommand{\braket}[2]{\left< #1 \vphantom{#2} \right|
	\left. #2 \vphantom{#1} \right>} % for Dirac brackets
\newcommand{\matrixel}[3]{\left< #1 \vphantom{#2#3} \right|
	#2 \left| #3 \vphantom{#1#2} \right>} % for Dirac matrix elements
\newcommand{\grad}[1]{\gv{\nabla} #1} % for gradient
\let\divsymb=\div % rename builtin command \div to \divsymb
\renewcommand{\div}[1]{\gv{\nabla} \cdot #1} % for divergence
\newcommand{\curl}[1]{\gv{\nabla} \times #1} % for curl
\let\baraccent=\= % rename builtin command \= to \baraccent
\renewcommand{\=}[1]{\stackrel{#1}{=}} % for putting numbers above =
\newtheorem{prop}{Proposition}
\newtheorem{thm}{Theorem}[section]
\newtheorem{lem}[thm]{Lemma}
\newtheorem{dfn}{Definition}


% BEGIN FUSION STATE DIAGRAMS

\makeatletter
\newcounter{mycounter}
\newcommand{\fs}[3][]{
	\begin{tikzpicture}[scale=0.3,font=\footnotesize,anchor=mid,baseline={([yshift=-.5ex]current bounding box.center)}]
	\def\height{1.5}
	\def\offset{0.5}
	\ifthenelse{\equal{#1}{}}{\def\height{1}}{} % Smaller height if no braiding
	\setcounter{mycounter}{0}
	\@for\el:=#2\do{
		\ifthenelse{\equal{#1}{\value{mycounter}}}{
			\braid at (\value{mycounter},\height) s_1^{-1};
			\stepcounter{mycounter}
		}{
			\stepcounter{mycounter}
			\ifthenelse{\equal{#1}{\value{mycounter}}}{
			}{
				\draw (\value{mycounter}, \height) to (\value{mycounter}, 0);
			}
		}
		\node at (\value{mycounter}, \height+\offset) {$\el$};
	}
	\draw (0, 0) to (\value{mycounter}+1, 0);
	\setcounter{mycounter}{0}
	\@for\el:=#3\do{
		\node at (\value{mycounter}+0.5, -0.6) {$\el$};
		\stepcounter{mycounter}
	}
	\end{tikzpicture}
}
\makeatother

\makeatletter
\newcommand{\fswide}[3][]{
	\begin{tikzpicture}[scale=0.3,font=\footnotesize,anchor=mid,baseline={([yshift=-.5ex]current bounding box.center)}]
	\def\height{1.75}
	\def\offset{0.5}
	\def\widthfactor{2}
	\ifthenelse{\equal{#1}{}}{\def\height{1}}{} % Smaller height if no braiding
	\setcounter{mycounter}{0}
	\@for\el:=#2\do{
		\ifthenelse{\equal{#1}{\value{mycounter}}}{
			\braid[width=\widthfactor cm, height=1.25 cm] at (\widthfactor*\value{mycounter},\height) s_1^{-1};
			\stepcounter{mycounter}
		}{
			\stepcounter{mycounter}
			\ifthenelse{\equal{#1}{\value{mycounter}}}{
			}{
				\draw (\widthfactor*\value{mycounter}, \height) to (\widthfactor*\value{mycounter}, 0);
			}
		}
		\node at (\widthfactor*\value{mycounter}, \height+\offset) {$\el$};
	}
	\draw (0, 0) to (\widthfactor*\value{mycounter}+\widthfactor*1, 0);
	\setcounter{mycounter}{0}
	\@for\el:=#3\do{
		\node at (\widthfactor*\value{mycounter} + \widthfactor*0.5, -0.6) {$\el$};
		\stepcounter{mycounter}
	}
	\end{tikzpicture}
}
\makeatother

\makeatletter
\newcommand{\fswideflex}[4][]{
	\begin{tikzpicture}[scale=0.3,font=\footnotesize,anchor=mid,baseline={([yshift=-.5ex]current bounding box.center)}]
	\def\height{1.75}
	\def\offset{0.5}
	\def\widthfactor{#4}
	\ifthenelse{\equal{#1}{}}{\def\height{1}}{} % Smaller height if no braiding
	\setcounter{mycounter}{0}
	\@for\el:=#2\do{
		\ifthenelse{\equal{#1}{\value{mycounter}}}{
			\braid[width=\widthfactor cm, height=1.25 cm] at (\widthfactor*\value{mycounter},\height) s_1^{-1};
			\stepcounter{mycounter}
		}{
			\stepcounter{mycounter}
			\ifthenelse{\equal{#1}{\value{mycounter}}}{
			}{
				\draw (\widthfactor*\value{mycounter}, \height) to (\widthfactor*\value{mycounter}, 0);
			}
		}
		\node at (\widthfactor*\value{mycounter}, \height+\offset) {$\el$};
	}
	\draw (0, 0) to (\widthfactor*\value{mycounter}+\widthfactor*1, 0);
	\setcounter{mycounter}{0}
	\@for\el:=#3\do{
		\node at (\widthfactor*\value{mycounter} + \widthfactor*0.5, -0.6) {$\el$};
		\stepcounter{mycounter}
	}
	\end{tikzpicture}
}
\makeatother

\makeatletter
\newcommand{\fswider}[3][]{
	\begin{tikzpicture}[scale=0.3,font=\footnotesize,anchor=mid,baseline={([yshift=-.5ex]current bounding box.center)}]
	\def\height{1.75}
	\def\offset{0.5}
	\def\widthfactor{3}
	\ifthenelse{\equal{#1}{}}{\def\height{1}}{} % Smaller height if no braiding
	\setcounter{mycounter}{0}
	\@for\el:=#2\do{
		\ifthenelse{\equal{#1}{\value{mycounter}}}{
			\braid[width=\widthfactor cm, height=1.25 cm] at (\widthfactor*\value{mycounter},\height) s_1^{-1};
			\stepcounter{mycounter}
		}{
			\stepcounter{mycounter}
			\ifthenelse{\equal{#1}{\value{mycounter}}}{
			}{
				\draw (\widthfactor*\value{mycounter}, \height) to (\widthfactor*\value{mycounter}, 0);
			}
		}
		\node at (\widthfactor*\value{mycounter}, \height+\offset) {$\el$};
	}
	\draw (0, 0) to (\widthfactor*\value{mycounter}+\widthfactor*1, 0);
	\setcounter{mycounter}{0}
	\@for\el:=#3\do{
		\node at (\widthfactor*\value{mycounter} + \widthfactor*0.5, -0.6) {$\el$};
		\stepcounter{mycounter}
	}
	\end{tikzpicture}
}
\makeatother

\newcommand{\fsfused}[5]{
	\begin{tikzpicture}[scale=0.3,font=\footnotesize,anchor=mid,baseline={([yshift=-.5ex]current bounding box.center)}]
	\def\height{1.75}
	\def\offset{0.5}
	\node at (1, \height+\offset) {$#2$};
	\node at (2, \height+\offset) {$#3$};
	\draw (0.5, 0) to (2.5, 0);
	\draw (1, \height) to [bend left=-30] (1.5, 1);
	\draw (2, \height) to [bend left=30] (1.5, 1);
	\draw (1.5, 1) to (1.5, 0);
	\node at (0.75, -0.6) {$#1$};
	\node at (2.25, -0.6) {$#4$};
	\node at (2, 0.7) {$#5$};
	\end{tikzpicture}
}

\newcommand{\fsfusedUp}[9]{
	\begin{tikzpicture}[scale=0.3,font=\footnotesize,anchor=mid,baseline={([yshift=-.5ex]current bounding box.center)}]
	\node at (0, 1.5+0.5) {$#1$};
	\node at (1, 2.5+0.5) {$#2$};
	\node at (2, 2.5+0.5) {$#3$};
	\node at (3, 1.5+0.5) {$#4$};
	\draw (-1, 0) to (4, 0); % horizontal
	\draw (1, 2.5) to [bend left=-30] (1.5, 1.5);
	\draw (2, 2.5) to [bend left=30]  (1.5, 1.5);
	\draw (1.5, 1.5) to (1.5, 0); % vertical under bend
	\draw (0,   1.5) to (0,   0); % vertical non-bend
	\draw (3,   1.5) to (3,   0); % vertical non-bend
	\node at (-0.75, -0.6) {$#5$};
	\node at (0.75, -0.6) {$#6$};
	\node at (2, 0.7) {$#7$};
	\node at (2.25, -0.6) {$#8$};
	\node at (3.75, -0.6) {$#9$};
	\end{tikzpicture}
}

\newcommand{\fsfuseddouble}[9]{
	\begin{tikzpicture}[scale=0.3,font=\footnotesize,anchor=mid,baseline={([yshift=-.5ex]current bounding box.center)}]
	\def\height{1.75}
	\def\offset{0.5}
	\node at (1, \height+\offset) {$#2$};
	\node at (2, \height+\offset) {$#3$};
	\node at (3, \height+\offset) {$#6$};
	\node at (4, \height+\offset) {$#7$};
	\draw (0.5, 0) to (4.5, 0);
	\draw (1, \height) to [bend left=-30] (1.5, 1);
	\draw (2, \height) to [bend left=30]  (1.5, 1);
	\draw (3, \height) to [bend left=-30] (3.5, 1);
	\draw (4, \height) to [bend left=30]  (3.5, 1);
	\draw (1.5, 1) to (1.5, 0);
	\draw (3.5, 1) to (3.5, 0);
	\node at (2,     0.5) {$#4$};
	\node at (0.75, -0.6) {$#1$};
	\node at (2.5,    -0.6) {$#5$};
	\node at (4,     0.5) {$#8$};
	\node at (4.25, -0.6) {$#9$};
	\end{tikzpicture}
}

\newcommand{\fsfusedbraided}[5]{
	\begin{tikzpicture}[scale=0.3,font=\footnotesize,anchor=mid,baseline={([yshift=-.5ex]current bounding box.center)}]
	\def\height{3}
	\def\offset{0.5}
	\node at (1, \height+\offset) {$#2$};
	\node at (2, \height+\offset) {$#3$};
	\braid at (1, \height) s_1^{-1};
	\draw (0.5, 0) to (2.5, 0);
	\draw (1, 1.5) to [bend left=-30] (1.5, 1);
	\draw (2, 1.5) to [bend left=30] (1.5, 1);
	\draw (1.5, 1) to (1.5, 0);
	\node at (0.5, -0.6) {$#1$};
	\node at (2.5, -0.6) {$#4$};
	\node at (2, 0.7) {$#5$};
	\end{tikzpicture}
}

% END FUSION STATE DIAGRAM


