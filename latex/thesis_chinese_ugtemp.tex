\documentclass[UTF8,12pt,a4paper]{report}
\usepackage{amsfonts}
\usepackage{graphicx}
\usepackage{ctex}
\usepackage[center]{titlesec}
\usepackage{amsmath} % AMS Math Package
\usepackage{amsthm} % Theorem Formatting
\usepackage{amssymb}	% Math symbols such as \mathbb
\usepackage{multicol} % Allows for multiple columns
\usepackage{cite}
\usepackage{float}
\usepackage{textcomp}
\usepackage[font=small,labelfont=bf]{caption}
\usepackage{setspace} 

\usepackage{titletoc}


\usepackage[
left=3.00cm,
right=2.5cm,
top=2.50cm,
bottom=2.00cm,
%headsep=0.25cm,
%headheight=1.5cm,
%footskip=0.79cm,
%	showframe%uncomment this if you want to check the page setup.
]{geometry}

\usepackage[super,sort&compress]{natbib}   %引用右上
\usepackage[titletoc]{appendix}
\usepackage{tikz}  % drawing
\usepackage{mathtools}

\linespread{1.18} %行距

%\usepackage{CJKutf8}
% Sets margins and page size
\makeatletter % Need for anything that contains an @ command 
\newcommand{\Fontxiaoer}{\fontsize{16pt}{\baselineskip}\selectfont} % 小二号
\newcommand{\Fontsan}{\fontsize{16pt}{\baselineskip}\selectfont} % 三号
\newcommand{\Fontsi}{\fontsize{14pt}{\baselineskip}\selectfont} %四号
\newcommand{\Fontxiaosi}{\fontsize{12pt}{\baselineskip}\selectfont}%小四
\newcommand{\Fontwu}{\fontsize{10.5pt}{\baselineskip}\selectfont}%五号



\renewcommand\bibname{\center参考文献}
\renewcommand{\figurename}{\Fontwu \bfseries 图}
\renewcommand{\contentsname}{\Fontxiaoer \heiti目\quad 录}
\renewcommand{\maketitle} % Redefine maketitle to conserve space
{ \begingroup  \begin{center} \large {\bf \@title}
		\vskip 10pt \large \@author \hskip 20pt \@date \end{center}
	\vskip 10pt \endgroup \setcounter{footnote}{0} }
\makeatother % End of region containing @ commands

\renewcommand{\labelenumi}{(\alph{enumi})} % Use letters for enumerate
% \DeclareMathOperator{\Sample}{Sample}

\newcommand{\abs}[1]{\left| #1 \right|} % for absolute value
\newcommand{\ket}[1]{\left| #1 \right>} % for Dirac bras
\newcommand{\bra}[1]{\left< #1 \right|} % for Dirac kets






%\usepackage[colorlinks,linkcolor=black,anchorcolor=black,citecolor=blue
%,CJKbookmarks=True,pdftex]{hyperref}
%如果文件采用了utf-8编码保存,那么千千万万不要在hyperref包的选项里面写入CJKbookmarks,这会导致乱码。而应该使用unicode选项。
%一个简单问题:如果采用pdflatex编译,选项中应该写pdftex;而如果采用类似latex->dvi2ps->ps2pdf这样的一条编译链的话就应该使用选项dvipdfm。

\usepackage[colorlinks,linkcolor=black,anchorcolor=black,citecolor=blue
,unicode=true,pdftex]{hyperref}


\setcounter{tocdepth}{2}   % 目录深度,只显示chap sec subsec


 %\renewcommand{\appendixname}{附录~\Alph{chapter}}

\begin{document}
\songti 

\titlespacing*{\chapter}{12pt}{0pt}{0pt}
\titleformat{\chapter}{\flushleft\Fontsan\heiti}{第\,\thechapter\,章}{1em}{}
\titlespacing*{\section}{0pt}{0pt}{0pt}
\titleformat{\section}{\flushleft\Fontsi\heiti}{\thesection}{1em}{}
\titlespacing*{\subsection}{0pt}{0pt}{0pt}
\titleformat{\subsection}{\flushleft\Fontxiaosi\heiti}{\thesubsection}{1em}{}
\titlespacing*{\subsubsection}{0pt}{0pt}{0pt}
\titleformat{\subsubsection}{\flushleft\Fontxiaosi\heiti}{\thesubsection}{1em}{}



\Fontsi{\tableofcontents}

\Fontxiaosi
\input{introduction}
\input{measure}












%---------------------------------------------附录
%\clearpage
%
%%\renewcommand{\thechapter}{\Alph{chapter}}
%\titleformat{\chapter}{\flushleft\Fontsan\heiti}{3}{1em}{}
%\chapter{附录}
%\input{supply}





%\input{superconductingqubit}{\tiny }
%\nocite{Bibtexkey}
% 文中未出现引用标记,但依然需要在参考文献中打印该文献时使用
\bibliographystyle{abbrv}% or plain
%plain,按字母的顺序排列,比较次序为作者、年度和标题.
%unsrt,样式同plain,只是按照引用的先后排序.
%alpha,用作者名首字母+年份后两位作标号,以字母顺序排序.
%abbrv,类似plain,将月份全拼改为缩写,更显紧凑.
%ieeetr,国际电气电子工程师协会期刊样式.
%acm,美国计算机学会期刊样式.
%siam,美国工业和应用数学学会期刊样式.
%apalike,美国心理学学会期刊样式.
\bibliography{thesis_bibli//thesis_bibli}
	
	
\end{document}
